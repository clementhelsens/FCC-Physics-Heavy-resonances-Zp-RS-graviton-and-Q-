%%%%%%%%%%%%%%%%%%%%%%%%%%%%%%%%%%%%%%%%%%%%%%%%%%%%%
\section{BSM-models}
\label{sec:physmodel}


In order to explore and contrast the capabilities of future colliders to discover and examine the properties of new physics, a broad set of benchmark models needs to be employed. In the 
case of new heavy resonances, this benchmark set should be sufficiently complete such that all of the major discovery channels of relevance are represented. As discussed above, here we 
are particularly interested in the 2-body final states of these resonances (since they are generally dominant in almost all new physics scenarios) consisting of opposite sign dilepton 
pairs ($e^+e^-, \mu^+\mu^-$ and $\tau^+\tau^-$), dijets, $ t\bar t$ and $W^+W^-$.  We note that it is highly likely that at least one or possibly more of these 2-body channels will posses 
a respectable branching fraction for the new resonances that result from any specific beyond the Standard Model scenario. Note that decays into pairs of secondary objects that 
then themselves decay hadronically can often populate the dijet channel if the final state jets are sufficiently boosted so this channel can represent many different final states unless 
substructure studies are performed.  When there are 2 or more of these channels available for simultaneous study we have an increased chance of learning significantly more about 
the underlying physics model behind the new resonance. The most important properties of a newly discovered resonance that need to be determined (other than the mass) are its 
production cross section, which will sometimes require a good understanding of the underlying background shape especially for a broad resonance and its spin (as was the case in 
the example of the Higgs boson). These properties alone can provide important information about the BSM model from which the signal originated. The spin measurement usually requires 
the reconstruction of the angular distribution of the resonance decay products and, hence, a respectable amount of statistics although the observation of certain final states can 
immediately exclude some spin possibilities as was the case with $H\rightarrow \gamma \gamma$.

A new, neutral, spin-1 gauge boson, $\Zp$, which is usually a color-singlet object produced in the $q\bar q$ channel, is a ubiquitous feature of many models that predict new 
physics~\cite{Langacker:2008yv,Rizzo:2006nw,Carena:2004xs,Salvioni:2009mt}.  While falling into several distinct classes, $\Zp$ are most commonly associated with the 
extension of the the SM electroweak gauge group by an additional U(1) or SU(2) factor although more significant augmentations are possible. When the additional factor is non-abelian, 
as in the case of SU(2), a new $W^\pm$' gauge boson generally also appears in the spectrum together with the $\Zp$ and with a comparable mass.  Of this subset of models, those 
that arise from Grand Unified Theory frameworks are the ones most commonly encountered in the literature and include familiar examples such as the Left-Right Symmetric 
Model (LRM)~\cite{Senjanovic:1975rk,Senjanovic:1975rk,Mohapatra:1980yp} which results from SO(10) 
(or larger GUT groups) and where the SM is augmented by an SU(2)$_R$ factor. Most simply, the LRM can arise, e.g., from SO(10) GUT breaking directly to 
SU(2)$_L \times$SU(2)$_R \times$U(1)$_{B-L}$ which then breaks to the SM at the few to multi-TeV scale. A second set of GUT-based $\Zp$ models arise from 
$E_6$~\cite{Robinett:1982tq,London:1986dk,Hewett:1988xc,Joglekar:2016yap} where most simply $E_6 \rightarrow$ SM$\times$U(1)$_\psi \times$U(1)$_\chi \rightarrow $ SM$
\times$U(1)$_\theta$, where a new U(1)$_\theta$ gauge group factor is predicted. Note here $\theta$ labels the remaining linear combination of U(1)$_\psi$-U(1)$_\chi$ that remains 
unbroken to lower energies (in comparison to the GUT scale).  A common set of features of this GUT-based model class include their possesing generation-universal couplings 
of the $\Zp$ to the SM fermions, their charges commuting with those of the SM so that, e.g., $u_L$ and $d_L$ have the same $\Zp$ coupling and the resonances themselves are usually 
narrow, reflecting electroweak strength or weaker couplings with width to mass ratios $\Gamma/M < 0.01-0.03$. In particular, the GUT origin of these models 
implies that this class of $\Zp$ can be used to simultaneously study all of the dileptonic channels: $e^+e^-,~\mu^+\mu^+$ as well as $\tau^+\tau^-$ together with the dijets and boosted 
$t\bar t$ channels as well.

With this much information potentially available from the observation of a given $\Zp$ in multiple channels one may try to distinguishable it from others of similar type given 
sufficient statistics and 
well-controlled systematics. In addition to relative cross section measurements,  e.g., that of dijets and/or $t\bar t$ compared to dileptons, the cleanliness of the dilepton channel itself 
can allow us to obtain additional information.  Since the leptons can be signed, their angular distribution allows us to determine their forward-backward asymmetry, $A_{FB}$, (defined in the 
dilepton center of mass frame) which depends upon the quark and lepton couplings of the $\Zp$ in a different manner than does the dilepton production cross section and, hence, will 
differ from model to model. Note that theoretically the scattering angle is usually defined as the one between the outgoing negatively charged lepton and the incoming valence quark direction which is {\it usually} also the direction of the boost of the center of mass frame as seen in the lab frame. However, sometimes this condition does not hold and the anti-quark direction is 
instead that of the boost and this must be corrected for statistically in Monte Carlo, but not an an event-by-event basis. This observable is discussed more fully below along with some of its 
alternative definitions. A second possibility~\cite{delAguila:1993ym} is to make use of the fact 
that the rapidity distributions of the $u\bar u$ and $d\bar d$ PDFs are somewhat different. Since various $\Zp$ will generally couple differently to the  $u$ and $d$ quarks the 
rapidity distributions of the dilepton final state will probe these coupling variations. This possibility can be probed by forming the rapidity ratio, $r_y$, which is the ratio of the number of 
central dilepton pairs to that at larger rapidities; this too will be discussed in further detail below. 

Returning to our discussion of these specific GUT-inspired models, we note that in the LRM with the assumption of left- and right-handed gauge couplings, i.e., $\kappa=g_R/g_L=1$, 
all of the various interactions of the $\Zp$ with the SM fields are completely fixed. However, in the $E_6$ model case, the 
single new mixing parameter, $\theta$, controls the couplings of the $\Zp$ to the various SM particles; four particular choices for the value of this parameter correspond to the more 
specific model cases discussed here and are denoted as $\psi$, $\chi$, $\eta$ and I. As in the SM, the $\Zp$ in GUT models generally couple to all the familiar quarks and leptons and 
thus can easily populate simultaneously the various fermionic 2-body final states listed above at various predictable rates. The measurement of these rates (as well as other associated 
observables) can be then used to discriminate among the various $\Zp$ possibilities after discovery as will be discussed further below. Note that the decay rate for $\Zp$ into the 
$W^+W^-$ final state in GUT frameworks is highly dependent on the details of the model building assumptions within a specific scenario and especially upon the detailed nature 
of spontaneous symmetry breaking as manifested by the amount of mixing (if it occurs at all) between the $\Zp$ and SM $\Z$; the $\Zp$ coupling to $W^+W^-$ in U(1) extensions is 
always controlled solely by the amount of this gauge boson mixing.  

The $\Zp$ of the Sequential Standard Model~\cite{Altarelli:1989ff} is often included within this GUT class of models although it is not a true gauge theory in the conventional sense. 
However, it has been used very frequently for many years as a standard candle by experimenters since it conveniently posits the existence of heavier copies of the usual SM gauge bosons 
with exactly the same couplings as do the gauge bosons in the SM; this provides a useful yardstick with which one can make comparisons easily. 

Alternative models of electroweak symmetry breaking, including the topcolor assisted technicolor scenarios, can also frequently lead to $\Zp$-like states~\cite{Hill:1994hp}
that can produce resonance signatures. The greatest difference of such theories from the GUT-type $\Zp$ model class lies in their having generation-dependent couplings of 
potentially QCD strength. (The color-octet versions of such states in this model class are called colorons.) This implies that the corresponding resonance will likely not be narrow and 
will preferentially couple, by construction, to the 
third generation, e.g., the highly boosted $t\bar t$ final state, thus proving another useful benchmark model for this channel. Similar new $\Zp$ states can also arise in Little Higgs 
models~\cite{ArkaniHamed:2001nc} which can also have preferential decays to third generation states. 

Occasionally the expected properties of a new $\Zp$ models are completely data-driven.  A $\Zp$ with an unusual flavor-dependent coupling structure has been suggested as 
a (partially complete) UV model to explain the apparent anomaly seen in semileptonic $b\rightarrow sl^+l^-$ decays~\cite{Aaij:2014ora,Aaij:2017vbb}. In effective field theory language, 
a new interaction of the form $\sim \bar bP_Ls \bar \mu P_L \mu$ of proper strength can provide a reasonable fit to these experimental observations~\cite{Bifani:2018zmi} which can 
be the result of the exchange of a 
very heavy $\Zp$ potentially accessible to high energy colliders ~\cite{Allanach:2017bta}. This $\Zp$, in the weak basis, couples only to the third generation quark doublet and to the 
muon lepton doublet so that it will have a suppressed production cross section at hadron colliders. Such a $\Zp$ could be observed in both the dimuon and ditop channels.

Models of composite quarks and leptons offer another path wherein new resonances are predicted. Excited quarks ~\cite{Baur:1987ga,Baur:1989kv}, $Q^*$, are spin-1/2, color triplet 
objects which carry the same SM quantum numbers as do the SM quarks. Here one imagines that the usual quarks have some type of internal structure and are held very tightly together 
by some new BSM force; these constituents when excited in some way yield the more massive states we would then observe as $Q^*$. There is, as of yet, no fundamental, UV-complete 
model encompassing this idea so that this framework is purely phenomenological. The SM quarks couple to these excited states via a magnetic dipole-like interaction together with an associated gauge boson such as the gluon or the SM $W,\Z$ or $\gamma$.  This interaction is suppressed by a large `compositeness scale', $\Lambda$, since it corresponds to a dim-6 operator, and the relative coupling strengths to the different gauge bosons are partially controlled by a set of essentially free parameters, $f_i$. Excited quarks can be singly produced in 
the $gq$ channel to which they will also dominantly decay due to the presence of the strong coupling constant, yielding the dijet signature of interest to us here although decays into, e.g, 
the $q\gamma$ channel are also of some interest. It is useful to have a benchmark model with dijet decays which take place in the $gq$ channel (as opposed to a $\Zp$ which can 
only populate the $q\bar q$ dijet channel) with which to compare and contrast. The angular distributions of the 2 jets in the dijet decay, which will require significant statistics to 
determine, can provide 
us information about the spin of the original resonance and the nature of its couplings to the decay products~\cite{Harris:2011bh,Boelaert:2009jm,Chivukula:2014pma,Chivukula:2017nvl}. 

Spin-2 graviton resonances occur in extra-dimensional scenarios that attempt to address the hierarchy problem, in particular, in the case of the warped extra dimensional model of 
Randall and Sundrum (RS)~\cite{Randall:1999ee}. In such setups, the SM gauge fields and fermions are generally allowed to propagate in the 5-D 
bulk~\cite{Pomarol:1999ad,Davoudiasl:1999tf,Grossman:1999ra,Davoudiasl:2000wi,Gherghetta:2000qt} whereas electroweak symmetry breaking occurs on or near the TeV/SM brane 
via the usual Higgs mechanism. This approach simultaneously helps to address the SM fermion mass hierarchy by the localization properties of the SM fermions in the bulk. 
One finds that, due to the shape of their 5-D wavefunctions, the Kaluza-Klein excitations of the familiar graviton, $G_{RS}$~\cite{Davoudiasl:1999jd} will dominantly decay into 
objects localized near to where SM symmetry breaking occurs, i.e., Higgs boson pairs and $t\bar t$ as well as to the longitudinal components of the massive SM gauge bosons, e.g., 
$W^+_L W^-_L$, all with relatively fixed branching factions with only some small allowed variations. Thus $G_{RS}\rightarrow 
W^+W^-$ in the RS framework provides an excellent benchmark model for the study of resonant $W$-pairs which are also quite highly boosted. If these $W$'s decay hadronically, 
given this high boost, this final state may also (appear to) populate the resonant dijet channel. One notes that apart from the $G_{RS}$ mass scale itself, essentially the only other 
free parameter in this RS model setup (wherein the lighter fermions are essentially decoupled from the graviton resonances), is frequently denoted by $c=k/\bar M_{Pl}$, which simply 
controls the overall coupling strength to all of the various SM particles. 







