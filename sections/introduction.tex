\section{Introduction}

Particle accelerators are built to answer some of the most fundamental questions about the natural world. For LHC it was guaranteed that the SM Higgs boson would be found, for the next generation of machines there is no possibilities to guarantee that any new particle will be discovered. Still with much higher center of mass energies compared to LHC, there are guaranteed deliverables like the study of the Higgs and top-quark properties and exploration of electroweak symmetry breaking phenomena with unmatchable precision and sensitivity. 

New machines are build to make direct discoveries, and even though we do not have any guarantee discoveries, we need to make sure that we will cover a large fraction of BSM phase space. So the mass reach should be increased by a factor of $\sqrt{s}/14$ so 7 for $\sqrt{s}=100$, but there are issues with parton luminosities evolving with $Q^2$, as at high energy they drop a bit faster, so typically it is a factor of 5 increase. But the statistics is enhanced by several orders of magnitudes for many BSM phenomena, that the LHC could barely touch during its exploitation. It is not only that the mass reach increases by a factor of five, but it?s the fact that if the LHC were to see some hints of possible new physics, by increasing the energy by a factor 7, we would increase the statistics by 2 or even 3 orders of magnitude, and we can use this new machine study with great accuracy what it is exactly. 
In addition we could have the ability to provide firm answers to questions like: is the SM dynamics all there at the TeV scale, is there a TeV scale solution to the hierarchy problem, is dark matter a thermal wimp (either we discover it as a WIMP, or we discover DM is not a WIMP and it has to be something else), was the cosmological EW phase transition 1st order?

Concerning the topic of this paper, the discovery potential of new resonances not predicted by the Standard Model (SM) makes Future Circular Colliders (FCC) to be the next  place to search for such heavy particles, compared to the current LHC and coming HL-LHC.
In the framework of FCC it is also extremely relevant to discuss the main limitations of the detector to identify high energetic top-quarks or W/Z bosons. Indeed 100\,TeV proton proton collisions will produce a very large amount of multi-TeV's bosons or top, the design of the detector needs detailed optimisations in order to achieve the required physics goals. The capabilities of such a detector should include the capabilities of measuring multi-TeV leptons, top-quarks and bosons, and will be discussed in this paper.

This document presents the expectations of some of the most relevant scenarios beyond the Standard Model (BSM) and the considered models are detailed in the section~\ref{sec:physmodel}.
The sample production, detector parametrisation, analysis statistical methods and other analysis techniques developed are presented in the section~\ref{sec:fccworkflow}.
The analyses strategy has been developed in the case of the FCC-hh scenario (section~\ref{sec:ana100tev}).
The leptonic resonances (ee, $\mu\mu$, $\tau\tau$) and the hadronic resonances (WW, $t\bar{t}$ and jj) analyses are detailed in sections~\ref{subsec:lepreso} and~\ref{subsec:hadreso} respectively.
But the results have also been extracted in the case of the HE-LHC scenario (section~\ref{sec:ana27tev}).
A study of model discrimination at HE-LHC in case of discovery at the end of HL-HLC is presented in section~\ref{sec:modeldiscri}.
