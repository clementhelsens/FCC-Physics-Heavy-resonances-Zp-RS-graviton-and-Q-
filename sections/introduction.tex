\section{Introduction}
The discovery potential of new resonances not predicted by the Standard Model (SM) and high or very high center of mass energy $\sqrt[]{s} = 27$~TeV (HE-LHC) or $100$~TeV (FCC-hh)
makes these machines to be the ultimate ones for such new particles, compared to the current LHC and coming HL-LHC.
\newline
The FCC-hh collider is also extremely relevant to discuss the main limitations of the detector to tag high energetic top-quarks or W/Z bosons.
The design of a 100 TeV proton-proton collider leads to many challenges for detector design. Detailed optimizations of the detector is needed in order to achieve the required physics goals.
The capabilities of such a detector should include the capabilities of measuring multi-TeV leptons, top-quarks and bosons.
\newline
\newline
This document presents the expectations of some of the most relevant scenarios beyond the Standard Model (BSM) and the considered models are detailed in the section~\ref{sec:physmodel}.
The sample production, detector parametrization, analysis limit settings and other analysis techniques developed are presented in the section~\ref{sec:fccworkflow}.
\newline
The analyses strategy has been developed in the case of the FCC-hh scenario (section~\ref{sec:ana100tev}).
The leptonic resonances (ee, $\mu\mu$, $\tau\tau$) and the hadronic resonances (WW, $t\bar{t}$ and jj) analyses are detailed in sections~\ref{subsec:lepreso} and~\ref{subsec:hadreso} respectively.
\newline
Then the results have been translated into the case of the HE-LHC scenario (section~\ref{sec:ana27tev}).
\newline
\newline
In the end, the study of model discrimination at HE-LHC is shown in section~\ref{sec:modeldiscri}.

